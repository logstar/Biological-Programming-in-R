\documentclass[12pt, t, xcolor=dvipsnames]{beamer}

% draft will print out comment
% final will hide all comment
\usepackage[draft]{pdfcomment}
\usepackage{xspace}
\usepackage{appendixnumberbeamer}
\usepackage{xcolor}
\usepackage{minted}
\usepackage[scale=2]{ccicons}
\usepackage{booktabs}
\usepackage{graphicx}
% \usepackage[absolute,overlay]{textpos}
\usepackage{pgfplots}
\usepackage{amsthm}
\usepackage{amsmath}


% \usepackage{hyperref}
% \includeonlyframes{current}

\newcommand{\pdfnote}[1]{\marginnote{\pdfcomment[icon=note]{#1}}}


\useoutertheme{metropolis}
\useinnertheme{metropolis}
\usefonttheme{metropolis}
\usecolortheme{metropolis}

\pgfplotsset{compat=1.14}

\newtheorem{mydef}{Definition}

\setbeamercolor{normal text}{fg=black!95, bg=white}
%\setbeamersize{text margin left=6mm, text margin right=6mm} 
\setlength{\leftmargini}{0pt}

\usemintedstyle{borland}

\graphicspath{ {figures/} }

\setbeamercolor{framesource}{fg=gray}
\setbeamerfont{framesource}{size=\tiny}

\makeatletter
\renewcommand{\footnoterule}{}
\define@key{beamerfootnote}{nonumber}[true]{\edef\beamer@footarg{0}\def\@makefnmark{}}% have to set a number in \beamer@footarg, then it won't be automatically generated one. setting \@makefnmark to be empty means the number isn't printed. but only use this in a group else it affects following footnotes!
% instead of 'nonumber', could just use 0 as an optional argument to \footnote, but OP reports that keyval complains about this in some situations
% http://tex.stackexchange.com/questions/89539
\newcommand{\source}[1]{{\footnote[nonumber]{%
    \begin{beamercolorbox}[right,wd=\dimexpr\hsize-1.8em\relax]{framesource}
      \usebeamerfont{framesource}\usebeamercolor[fg]{framesource} Source: {#1}
    \end{beamercolorbox}}}}
\setlength{\footnotesep}{0cm}%\footnotesep is the space between footnotes (generated with a \rule)
\makeatother

% inspired by [beamer: footnote text collides with navigation symbols](http://tex.stackexchange.com/a/5855)
\addtobeamertemplate{sidebar right}{\setbeamertemplate{navigation symbols}{}}{}
\addtobeamertemplate{footline}{\hfill\usebeamertemplate***{navigation symbols}%
    \hspace*{0.1cm}\par\vskip 2pt}{}

% \setbeamercolor{framesource}{fg=gray}
% \setbeamerfont{framesource}{size=\tiny}
% 
% \newcommand{\source}[1]{\begin{textblock*}{4cm}(8.7cm,8.6cm)
%     \begin{beamercolorbox}[ht=0cm,right]{framesource}
%         \usebeamerfont{framesource}\usebeamercolor[fg]{framesource} Source: {#1}
%     \end{beamercolorbox}
% \end{textblock*}}

\tikzset{
  every overlay node/.style={
    draw=black,fill=white,rounded corners,anchor=north west,
  },
}

\def\tikzoverlay{%
   \tikz[baseline,overlay]\node[every overlay node]
}%

\definecolor{codegray}{gray}{0.95}
\newcommand{\code}[1]{\colorbox{codegray}{\textcolor{black!95}{\texttt{#1}}}}
%\newcommand{\code}[1]{\texttt{#1}}}

\newminted{R}{fontsize=\scriptsize, 
              frame=lines,
              bgcolor=codegray,
              framesep=1mm}

\newcommand {\framedgraphic}[2] {
    \begin{frame}{#1}
        \begin{center}
            \includegraphics[width=\textwidth,height=0.8\textheight,keepaspectratio]{#2}
        \end{center}
    \end{frame}
}

\pgfmathdeclarefunction{gauss}{3}{%
  \pgfmathparse{1/(#3*sqrt(2*pi))*exp(-((#1-#2)^2)/(2*#3^2))}%
}

\title{Week 3 Lab}
% \subtitle{}
\date{Tuesday, January 30/Thursday, February 1}
%\author{Yuanchao Zhang}
% \institute{}
% \titlegraphic{\hfill\includegraphics[height=1.5cm]{logo.pdf}}
\renewcommand\appendixname{Appendix}

\begin{document}

\maketitle

\begin{frame}{Plan for today}
\setbeamertemplate{section in toc}[sections numbered]
\begin{minipage}[t][3cm][t]{\textwidth}
  \tableofcontents
\end{minipage}

\end{frame}


\section{Changes of lab session format}

\begin{frame}{Changes of lab session administration}
  \begin{itemize}
    \item Lab handout:
    \begin{itemize}
       \item Include "To Think About" questions
     \end{itemize}
    \item Lab format:
    \begin{itemize}
       \item More similar to benchwork lab, e.g. chemistry
       \item Less lecturing
       \item More reading and practicing
     \end{itemize}
  \end{itemize}
  \pdfnote{The instructors will make sure that the lab topics will all be covered in the lecture. }
\end{frame}

\begin{frame}{If you were confused by the previous lab format...}
\begin{itemize}
  \item Sorry. We (TAs) did not clearly explain the design. 
  
  \item The previous lab sessions were designed to \alert{prepare} you for reading the write-up independently:

  \begin{itemize}
    \item Review important theoretical points in the lecture
    \item Explore programming concepts interactively (more time for questions)
    \item Make sure that the code can be executed on you computer
    \item Generally explain the design of code
    \item Give personal suggestions on controversial topics
    \item Leave the reading after the lab session at your own pace
    \item Leave trouble-shooting to Canvas or office hour
    \item Leave practicing to homework assignments
  \end{itemize}

\end{itemize}

\pdfnote{T1: Generally more helpful to have two ways of explaining the same concept.}

\pdfnote{T2: Get answers to your questions on certain concepts.}

\pdfnote{T3: So that you will not waste time on weird errors when reading the write-up. }

\pdfnote{T4: Explain why we chose this way to implement an idea. }

\pdfnote{T6: You all have your own pace of reading. }

\end{frame}

\begin{frame}{If you were confused by the previous homework...}
\begin{itemize}
  \item Grading:
  \begin{itemize}
    \item We are not expecting you to give us a single "correct" answer.
    \item We are expecting you to explain your reasoning in your answer.
    \item We will accept all reasonable answers with important reasoning steps.
    \item However, do not think too far away. There is always a solution using the lecture and lab materials.
    \item If you are thinking about methods not covered in the lecture or lab, discuss with the TAs or instructors.
  \end{itemize}
\end{itemize}
\end{frame}

\begin{frame}{If you were confused by the previous homework...}
\begin{itemize}
  \item You can:
  \begin{itemize}
    \item Start early. 
    \item Get help during office hour or through Canvas. 
  \end{itemize}
  \item We will:
  \begin{itemize}
    \item The instructors will try to put the homework quesetions in a more controlled context. 
    \item The instructors will try to ask more natural questions. 
    \item We TAs will try to discuss possible confusions about the homework at the end of the lab. 
  \end{itemize}
\end{itemize}
\end{frame}


\section*{Recap}

\begin{frame}{Recap}
  \begin{itemize}
    \item Different interpretations of random variable.
    \item The normal distribution.
    \item Testing for normality:
    \begin{itemize}
      \item Plotting: Histogram, density plot, and Q-Q plot.
      \item Statistical testing: Shapiro-Wilk normality test and Anderson-Darling normality test
    \end{itemize}
    \item The central limit theorem.
    \item One-sample tests of proportion
    \begin{itemize}
      \item z-test and $\chi^2$-test
      \item One-sided versus two-sided
      \item Power analysis
    \end{itemize}
  \end{itemize}
\end{frame}

\section{Bernoulli distribution} % (fold)
\label{sec:bernoulli_distribution}

\begin{frame}{Bernoulli distribution}
Notation: $X \sim \text{Bern}(p)$

Parameter: $p \in [0, 1]$ -- success probability of a Bernoulli trial.

Support: $x \in \{0, 1\}$

Probability mass function: \begin{equation*}
  f(x; p)=\begin{cases}
    p, & \text{for $x = 1$}.\\
    q = (1-p), & \text{for x = 0}.
  \end{cases}
\end{equation*}

Bernoulli distribution is a special case of binomial distribution. 

R has \code{rbinom()} but not \code{rbern()}. 

\pdfnote{More fun in binom. }
\source{\url{https://en.wikipedia.org/wiki/Bernoulli_distribution}}
\source{\url{https://en.wikipedia.org/wiki/Bernoulli_trial}}
\end{frame}
% section bernoulli_distribution (end)



\section{Binomial Coefficient} % (fold)
\label{sec:binomial_coefficient}

\begin{frame}{Binomial Coefficient}

\begin{theorem}[Binomial theorem]
$$ (x+y)^n = \sum_{k=0}^n \binom{n}{k} x^{n - k} y^k $$
\end{theorem}

\begin{mydef}[Binomial coefficient]
Any of the positive integers that occurs as a coefficient in the binomial theorem is a binomial coefficient.
\end{mydef}

Notation: $$\binom{n}{k} = \frac{ n! }{ k!(n-k)! }$$

\source{\url{https://en.wikipedia.org/wiki/Binomial_theorem}}
\source{\url{https://en.wikipedia.org/wiki/Binomial_coefficient}}

\end{frame}
% section binomial_coefficient (end)

\section{Binomial Distribution} % (fold)
\label{sec:binomial_distribution}

\begin{frame}{Binomial Distribution}
Intuition: the distribution of the sum of $n$ random variables $i.i.d.$ $\text{Bern}(p)$.

$$X \sim B(n, p)$$

\begin{itemize}
  \item Parameters:
  \begin{itemize}
    \item $n \in N_0$ -- number of trials
    \item $p \in [0, 1]$ -- success probability in each trial
  \end{itemize}
  \item Support: $x \in \{0, 1, 2, ..., n\}$
  \item Probability mass function: $$f(x; n, p) = \binom{n}{k} \cdot p^k \cdot (1-p)^{n-k}$$
\end{itemize}

\source{\url{https://en.wikipedia.org/wiki/Binomial_distribution}}
\end{frame}

% section binomial_distribution (end)

\section{Confidence Intervals} % (fold)
\label{sec:confidence_intervals}

\begin{frame}{Confidence interval}
\begin{itemize}
  \item Intuition: express the confidence of estimating population parameter using a sample of observations as an interval of estimates.
  \item Straightforward in algebraical definitions:
  \begin{itemize}
    \item Closely related to hypothesis testing.
    \item Multiple methods to calculate the confidence intervals (CIs) of the same population parameter. 
  \end{itemize}
  
\end{itemize}
\end{frame}

\begin{frame}{Controversy of interpreting confidence interval}
\begin{itemize}
  \item Controversial interpretations of confidence level (described without mathematical rigor):
  \begin{itemize}
    \item Apply algebraical equivalence: probability that the population parameter lies in the interval.
    \item Apply frequentist claimed definition: the proportion of intervals containing the population parametres in a large number of samples. 
  \end{itemize}
  \item The controversy lies in the philosophical question of whether the population parameter is random (algebraical interpretation) or constant (frequentist interpretation). 
  \item The algebraical interpretation is usually referred as "\alert{misinterpretation}" or "incorrect".
\end{itemize}
\vspace{-3mm}
\source{\url{https://onlinecourses.science.psu.edu/stat414/node/197}}
\source{\href{https://stats.stackexchange.com/questions/6652/what-precisely-is-a-confidence-interval}{whuber at StatsExchange}}
\end{frame}

\begin{frame}{Example: constant or random population parameter}
\begin{itemize}
  \item Scenario: estimate the mean effect of a drug on a disease measured as continuous value with arbitrary unit in $\mathbb{R}$.
  \item Population parameter: the mean effect of the drug, $\mu$.
  \item Categorization (determined using domain specific knowledge): for the same disease, a drug is good if its $\mu \geq 100$, otherwise bad.
  \item Sample: 500 patients. Mean 110. 95\% confidence interval of population mean: $[100, 120]$.
  \item Which claim on the drug do you prefer?
  \begin{itemize}
    \item Significantly good, $p \leq 0.05$. (constant)
    \item 97.5\% chance to be good. (random)
  \end{itemize}
\end{itemize}

\end{frame}

% section confidence_intervals (end)

\section{One sample Student’s t-test} % (fold)
\label{sec:One sample Student’s t-test}

% section section_name (end)
\begin{frame}{One sample Student’s t-test}
Intuition from CI perspective: calculate CI for population mean without requiring large sample size or known population standard deviation.

\begin{theorem}
If $X_1, X_2, ..., X_n$ are normally distributed random variables with mean $\mu$ and variance $\sigma^2$, then a confidence interval of confidence level $1 -− \alpha$ for the population mean $\mu$ is: $$ \bar{x} \pm t_{\alpha/2, n-1} (\frac{s}{\sqrt{n}})$$
\end{theorem}

\source{\url{https://onlinecourses.science.psu.edu/stat414/node/199}}
\end{frame}

\section{One sample Wilcoxon Signed-Rank test} % (fold)
\label{sec:one_sample_wilcoxon_signed_rank_test}

% section one_sample_wilcoxon_signed_rank_test (end)
\begin{frame}{One sample Wilcoxon Signed-Rank test}
\begin{itemize}
  \item Null hypothesis ($H_0$): the distribution of a sample is symmetric about $\mu_0$.

  \item Non-parametric test procedure: no known model for data points. 

  \item Intuition: 
  \begin{itemize}
    \item Use rank to ignore the magnitude of differences between different data points.
    \item Use sign, whether greater than or less than $\mu_0$, to describe the "direction" of the data point relative to $\mu_0$
    \item If $H_0$ is true, the sum of signed rank follows certain distribution.
  \end{itemize}
\end{itemize}
\end{frame}

% section significance_tests (end)

\section{Power analysis for one-sample t-test} % (fold)
\label{sec:power_analysis_for_one_sample_t_test}

% section power_analysis_for_one_sample_t_test (end)

\begin{frame}{General procedure of power analysis}
\begin{itemize}
  \item Calculation of power: usually before having the data, if the \alert{alternative hypothesis $H_1$} is true, how likely can we \alert{reject the null hypothesis $H_0$} with significance level \alert{$\alpha$}. 
  \item Parameters:
  \begin{itemize}
    \item Arbitrary constants: $H_1$, $H_0$, and $\alpha$.
    \item Constant depend on $H_1$ and $H_0$: $effect size$. 
    \item Variables: power and sample size $n$. 
    \item A priori power analysis: set power and calculate $n$. 
    \item Post hoc power analysis: set $n$ and calculate power. 
  \end{itemize}
  \item By convention from Wikipedia, power is set to $0.8$ when calculating sample size.
\end{itemize}
\end{frame}

\begin{frame}{Intuition of power analysis}
\begin{itemize}
  \item Good intuition of power analysis is unknown to me. 
  \item Best I can get is $\text{power} = 1 - \beta$.
  \item Why? No generalized mathematical framework for calculating power comparing to hypothesis testing. 
  \item Better to reason through the calculation of each specific case using the definition: $$\text{power} = \Pr {\big (}{\text{reject }}H_{0}\mid H_{1}{\text{ is true}}{\big )}$$
\end{itemize}
\end{frame}

\begin{frame}{Power analysis for one-sample t-test}
\begin{itemize}
  \item $H_0$: population mean $\mu = \mu_0$.
  \item $\alpha = 0.05$.
  \item Set $n$ or power.
  \item Very complicated in deriving the mathematical formula of $\text{power} = \Pr {\big (}{\text{reject }}H_{0}\mid H_{1}{\text{ is true}}{\big )}$. 
  \begin{itemize}
    \item Various $H_1$s: $\mu = \mu_1$, or $\mu > \mu_0$, or $\mu < \mu_0$, or $\mu \neq \mu_0$.
    \item Different $H_1$s may result in different formulas.
  \end{itemize}
  \item Very complicated interpretation of the effect size.
  \begin{itemize}
    \item Assumes population variance takes certain value.
    \item Possibly various calculation methods. 
  \end{itemize}
  
  \item Call \code{pwr.t.test} to avoid complexities, and check the mathematics when necessary (usually not). 
\end{itemize}

\end{frame}


% section power_analysis (end)

\begin{frame}[c]
  \large{Questions?}
\end{frame}


\end{document}
